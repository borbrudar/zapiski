\documentclass{report}

\input{common/preamble}
\input{common/macros}
\input{common/letterfonts}

\title{\Huge{Logika in mnozice}}
\author{Bor Brudar}
\date{}

\begin{document}
\maketitle
\newpage% or \cleardoublepage
% \pdfbookmark[<level>]{<title>}{<dest>}
\pdfbookmark[section]{\contentsname}{toc}
\tableofcontents
\pagebreak


\chapter{Mnozice in preslikave}

\section{}

\dfn{Princip ekstenzialnosti (Prvi aksiom) teorije mnozic}{
	Mnozici $X$ in $Y$ sta enaki natanko tedaj, ko vsebujeta iste elemente.
}

\clm{}{}{
	Naj bo $A \defeq \{ 1,3,5 \}$ in $B \defeq {5,3,5,1,3,5}$. $A$ in $B$ sta enaki.
}

\pf{Dokaz}{
	Vsak element mnozice $A$ je element mnozice  $B$ in obratno.

}


\end{document}
