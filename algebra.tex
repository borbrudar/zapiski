\documentclass{report}

\input{common/preamble}
\input{common/macros}
\input{common/letterfonts}

\title{\Huge{Algebra}}
\author{Bor Brudar}
\date{}

\begin{document}
\maketitle
\newpage% or \cleardoublepage
% \pdfbookmark[<level>]{<title>}{<dest>}
\pdfbookmark[section]{\contentsname}{toc}
\tableofcontents
\pagebreak


\chapter{}
\section{Osnovne}

\dfn{Linearna Kombinacija}{
	Naj bodo $\vec{a_1},\vec{a_2},\ldots,\vec{a_n}$ poljubni vektorji.
	Vsak vektor oblike $\alpha_1 \vec{a_1} + \alpha_2 \vec{a_2} \ldots $, kjer so $\alpha_1 \ldots \in \R $, imenujemo linearna kombinacija vektorjev $\vec{a_1} \ldots \vec{a_n}$.
	Primer: $2 \vec{a_1} + \frac{3}{2} \vec{a_2} - \vec{a_3}$ je linearna kombinacija vektorjev $\vec{a_1},\vec{a_2},\vec{a_3}$.
}

\dfn{Linearna odvisnost}{
	Vektorji $\vec{a_1}, \vec{a_2}, \ldots, \vec{a_n}$ so linearno odvisni, kadar lahko enega od njih izrazimo kot linearno kombinacijo ostalih. Vektorji, ki niso linearno odvisni, so linearno neodvisni.
}

Kadar sta 2 vektorja $\vec{a}$ in $\vec{b}$ linearno odvisna, velja $\vec{b} = \alpha \vec{a}$ ali $\vec{a} = \beta \vec{b}$. $\vec{a}$ in $\vec{b}$ sta vzporedna (t.j. vsaka usmerjena daljica, ki doloca $\vec{a}$, je vzporedna vsaki usmerjeni daljici, ki doloca $\vec{b}$. (Ko krajevna vektorja $\vec{a}$ in $\vec{b}$ (skozi izhodisce) lezita na isti premici).


Ce sta $\vec{a}$ in $\vec{b}$ linearno neodvisna, potem vektor $\alpha \vec{a} + \beta \vec{b}$ lezi v ravnini, ki jo dolocata ta dva krajevna vektorja.

\qs{}{Velja tudi obratno? Je vsak vektor iz te ravnine oblike $\alpha \vec{a} + \beta \vec{b}$?}

\sol Skozi koncno tocko vektorja $\vec{r}$ potegnemo vzporednico $\vec{b}$. Ker sta $\vec{a}$ in $\vec{b}$ linearno neodvisna, to ta premica seka premico, ki jo doloca $\vec{a}$ v natanko eni tocki. Vektor od $\vec{0}$ do te tocke je $\alpha \vec{a}$ za $\alpha \in \R$. Vektor od te tocke do konca $\vec{r}$ je oblike $ \beta \vec{b} ,\beta \in \R$. Od tod sledi $\vec{r} = \alpha \vec{a} + \beta \vec{b}$.


Dokazali smo, da je ravnina napeta med $\vec{a}$ in $\vec{b}$ mnozica vseh vektorjev oblike $\alpha \vec{a} + \beta \vec{b} \text{,kjer sta } \alpha,\beta \in \R $.


\cor{}{Trije krajevni vektorji so linearno odvisni natanko tedaj, ko lezijo na isti ravnini.}

\dfn{Baza prostora}{
	Trije linearno odvisni vekotrji so baza prostora $\R^3$. (Baza prostora $\R^3$ je mnozica, ki jo sestavljajo 3 linearno neodvisni vektorji.)
}


\nt{ Baza ravnine $\R^2$ je mnozica, ki jo sestavljata 2 linearno neodvisna vektorja.}


\clm{}{}{
	Naj bo ${ \vec{a}, \vec{b}, \vec{c} }$ baza prostora $\R^3$. Vsak vektor $\vec{r} \in \R^3$ lahko zapisemo v obliki $\vec{r} = \alpha \vec{a} + \beta \vec{b} + \gamma \vec{c} ; \alpha, \beta, \gamma \in \R^3$. Tak zapis je enolicen.
}

\pf{Dokaz: (obstoj in enolicnost)}{
	Lahko predpostavimo, da so $\vec{a}, \vec{b}, \vec{c}, \vec{r}$ krajevni vektorji.

	Potegnemo premico, ki je vzporedna $\vec{c}$ in poteka skozi koncno tocko $\vec{r}$. Ta premica seka ravnino, ki jo dolocata $\vec{a}$ in $\vec{b}$ v natanko eni tocki. Ker ta tocka lezi na ravnini, je vektor od $\vec{c}$ do te tocke $\alpha \vec{a} + \beta \vec{b} ; \alpha, \beta \in \R$, kot smo dokazali prej. Vektor od te tocke do koncne tocke $\vec{r}$ je $\gamma \vec{c}, \gamma \in \R \implies \vec{r} = \alpha \vec{a} + \beta \vec{b} + \gamma \vec{c}$.


	Enolicnost: Recimo, da je $\vec{r} = \alpha \vec{a} + \beta \vec{b} + \gamma \vec{c} = \alpha^{\prime} \vec{a} + \beta^{\prime} \vec{b} + \gamma^{\prime} \vec{c} \\
	= (\alpha - \alpha^{\prime}) \vec{a} + (\beta - \beta^{\prime}) \vec{b} + (\gamma - \gamma^{\prime}) \vec{c} = \vec{0}.$

Ce je npr. $\alpha \neq \alpha^{\prime}$ lahko izrazimo $\vec{a}$ kot linearno kombinacijo $\vec{b}$  in $\vec{c}$:

\[
\vec{a} = - \frac{\beta - \beta^{\prime}}{\alpha - \alpha^{\prime}} \vec{b}- \frac{\gamma - \gamma^{\prime}}{ \alpha - \alpha^{\prime}} \vec{c}
.\]

Kar je v protislovju z linerano neodvisnostjo $\vec{a},\vec{b}, \vec{c}$. Sledi da $\alpha = \alpha^{\prime}$. Enako za $\beta = \beta^{\prime}$ in $\gamma = \gamma^{\prime}$.
}

\cor{}{Stirje vektorji v $\R^3$ so vedno linearno odvisni.}

\ex{Primer baze}{
	Definirajmo $ \vec{i} = (1,0,0), \vec{j} = (0,1,0)$ in $\vec{k} = (0,0,1)$. Mnozica ${ \vec{i}, \vec{j}, \vec{k}}$ je baza prostora $\R^3$,ki ji pravimo standardna baza.

	Ce $(x,y,z) \in \R^3$, ga razvijemo po standardni bazi tako: \[
		(x,y,z) = x \vec{i} + y \vec{j} + z \vec{k}
	.\]
}

\nt{
	Standardna baza v $\R^2$ je ${\vec{i},\vec{j}}$. Kjer je $
	\vec{i} = (1,0), \vec{j} = (0,1)$.
}


\clm{}{}{
	Vektorji $\vec{a},\vec{b},\vec{c} \in \R^3$ so linearno neodvisni natanko takrat, ko velja naslednje:

	Ce je $\alpha \vec{a} + \beta \vec{b} + \gamma \vec{c} = \vec{0}$, potem je $\alpha = \beta = \gamma = 0.$
}

\nt{
	Ce $\alpha = \beta = \gamma = 0$, potem je $
	\alpha \vec{a} + \beta \vec{b} + \gamma \vec{c} = \vec{0}$.
	To je obratno od trditve. Ce so linerano odvisne, so lahko $\vec{0}$, tudi ce niso vsi $\alpha, \beta, \gamma = 0$.
}

\pf{Dokaz}{
	($\implies$)
	Predpostavimo, da so $\vec{a}, \vec{b}, \vec{c}$ linearno neodvisni.


	Recimo, da velja $\alpha \vec{a} + \beta \vec{b} + \gamma \vec{c} = \vec{0}$ za $ \alpha, \beta, \gamma \in \R$.


	Ce je $\alpha \neq 0$, lahko $\vec{a} = - \frac{\beta}{\alpha} \vec{b} - \frac{\gamma}{\alpha} \vec{c}$, kar je v protislovju z linerano neodvisnostjo $\vec{a},\vec{b},\vec{c}$. Isto za $\beta, \gamma$.

	($\impliedby$) Predpostavimo, da velja sklep, da iz $\alpha \vec{a} + \beta \vec{b} + \gamma \vec{c} = \vec{0}$. Sledi $\alpha = \beta = \gamma = 0$. Dokazujemo da so $\vec{a}, \vec{b}, \vec{c}$ linearno neodvisni.

	Recimo, da niso : potem je eden od njih linearna kombinacija ostalih dveh. Predpostavimo lahko, da je to $\vec{a}$.

	\[
	\vec{a} = \beta \vec{b} + \gamma \vec{c} , \beta,\gamma \in \R \\
	\implies - \vec{a} + \beta \vec{b} + \gamma \vec{c} = 0
	.\]

	To je v nasprotju z zacetno predpostavko. $\vec{a},\vec{b},\vec{c}$ so torej linearno neodvisni.
}


\section{Skalarni produkt}

\dfn{Skalarni produkt vektorjev}{

	Skalarni produkt vektorjev
	$\vec{a_1} = (x_1, y_1, y_1)$ in $\vec{a_2} = (x_2, y_2, z_2)$ je
	stevilo $\vec{a_1} \vec{a_2} = x_1 x_2 + y_1 y_2 + z_1 z_2$.
}


Lastnosti skalarnega produkta:

\begin{enumerate}
	\item Komutativnost: $\vec{a} \vec{b} = \vec{b} \vec{a}; \forall \vec{a}, \vec{b} \in \R^3$
	\item Distributivnost: $\vec{a}(\vec{b} + \vec{c}) = \vec{a}\vec{b} + \vec{a} \vec{c} ; \forall \vec{c},\vec{b} \in \R^3$
	\item Homogenost: $ (\alpha \vec{a}) \vec{b} = \alpha(\vec{a} \vec{b}) = \beta(\alpha \vec{b}) ; \alpha \in \R$
	\item Pozitivna definitnost: $\vec{a} \vec{a} = 0 ,\forall \vec{a} \in \R^3$. In pa $\vec{a} \vec{a} = 0$, kadar $\vec{a} = \vec{0}$
\end{enumerate}

Asociativnost je neumnost: $(\vec{a} \vec{b}) \vec{c} \neq \vec{a} (\vec{b} \vec{c})$. (Prvi oklepaj je vzporeden $\vec{c}$, drugi pa vzporeden $\vec{a}$.


Lastnosti dokazemo tako, da vektorje napisemo po komponentah in poracunamo obe strani enakosti.

Npr. (4) : Naj bo $\vec{a} = (x,y,z)$

\[
\vec{a}\vec{a} = x x + y y + z z = x^2 + y^{2} + z^{2} \ge 0
.\]

(1) do (4) so aksiomi za skalarni produkt, ko obravnavamo vektorske prostore s skalarnim produktom.

\dfn{Dolzina (norma) vektorja}{
	Dolzina (norma) vektorja $\vec{a} \in \R^3$ je stevilo $\|\vec{a}\| = \sqrt{\vec{a}\vec{a}} $.
}


Ce $\vec{a} = (x,y,z) $ krajevni vektor je $\|\vec{a}\| = \sqrt{x^{2} + y^{2} + z^{2}} $, kar je dolzina usmerjene daljice, ki pripada temu vektorju.


Dolzina vektorja je torej enaka dolzini vsake usmerejene daljice, ki pripada temu vektorju.

\thm{}{

	\[ \vec{a} \vec{b} = \|\vec{a}\| \|\vec{b}\| \cos \phi,\]
	$\phi$ je kot (med $0$ in $\pi$) med usmerjenima daljicama s skupnim zacetkom.
}

\pf{Dokaz}{

	Slikca manka lol

	Kosinusni izrek: $\|\vec{b} - \vec{a}\| = \|\vec{b}\|^{2} + \|\vec{a}\|^{2} - 2 \|\vec{a}\| \|\vec{b}\| \cos \phi$.

	Upostevamo, da je $\|\vec{b} - \vec{a}\|^{2} = (\vec{b} - \vec{a}) (\vec{b} - \vec{c})$ in to poracunamo po distributivnosti.

	\[
	\vec{b} \vec{b} - \vec{a} \vec{b} - \vec{b} \vec{a} + \vec{a} \vec{a} = \vec{b} \vec{b} + \vec{a} \vec{a} - 2 \|\vec{a}\| \|\vec{b}\| \cos \phi
	\]

	\[
-2 \vec{a} \vec{b} \implies \vec{a} \vec{b} = \|\vec{a}\| \|\vec{b}\| \cos \phi.\]
}


Dokaz kosinusnega izreka brez skalarnega produkta: (slika lol)


Pitagorov izrek :
\[a^{2} = (b \sin \alpha)^{2} + (c -\cos \alpha)^{2} = b^{2}\sin^{2}\alpha + c^{2} - 2bc \cos \alpha + b^{2} \cos^{2} \alpha) = b^{2} + c^{2} - 2bc \cos \alpha.\]

Razmislek za topokotni trikotnik??


Dokaz Pitagorejskega izreka brez kosinusnega izreka in brez vektorjev. (spet slika)

\[
P = (a+b)^{2} = c^{2} + 4 \frac{ab}{2} = a^{2} + b^{2} + 2ab = c^{2} + 2ab = a^{2} + b^{2} = c^{2}
.\]

Dogovorimo se, da je $\vec{0}$ pravokoten na vsak vektor.

\cor{}{
\[
\vec{a} \vec{b} = 0 \iff \vec{a} \perp \vec{b}
.\]
}
\ex{}{
	\begin{enumerate}
		\item
	\[
	\vec{i} \vec{j} = \vec{i} \vec{k} = \vec{j} \vec{k} = 0
	.\]
	Standardni bazni vektorji so si med sabo pravokotni.

	\item Izracunaj kot med $\vec{a} = (1,1,2)$ in $\vec{b} = (1,0,1)$

	\[
	\vec{a} \vec{b}  = \|\vec{a}\| \|\vec{b}\| \cos \phi
	.\]
	\[
	\cos \phi = \frac{\vec{a} \vec{b}}{\|\vec{a}\| \|\vec{b}\|} = \frac{1 1 + 1 0 + 2 1}{\sqrt{6} + \sqrt{2}}  = \frac{3}{2\sqrt{3} }
	.\]


	\item $\vec{a_1} = (x_1,y_1,0)$ in $\vec{a_2} = (x_2,y_2,0)$ vektorja v ravnini $z=0$. Z $\vec{a_1}$ in $\vec{a_2}$ izrazi ploscino P paralelograma vpetega med $\vec{a_1}$ in $\vec{a_2}$.

		(slikca)

		Naj po $\phi$ kot med $\vec{a_1}$ in $\vec{a_2}$. Potem $\phi = \|\vec{a_1}\| \|\vec{a_2}\| \sin \phi$.

		Predpostavimo da je par $\vec{a_1},\vec{a_2}$ pozitivno orientiran. Vektor $\vec{a_1}$ zavrtimo za $\frac{\pi}{2}$ v + smer. Dobljeni vektor ke $\vec{a_1}^{\prime}$ in naj bo $\psi$ kot med  $\vec{a_1}^{\prime}$ in $\vec{a_2}$.

		\[\psi = \begin{cases}
			\frac{\pi}{2} - \phi ;\phi \le \frac{\pi}{2} \text{V obeh primerih je $\cos \psi = \sin \phi$} \\
			\phi - \frac{\pi}{2}; \phi > \frac{\pi}{2}
		\end{cases}\]

		\[\implies P = \|\vec{a_1}\| \|\vec{a_2}\| \sin \phi = \|\vec{a_1}\| \|\vec{a_2}\| \cos \phi = \vec{a_1}^{\prime} \vec{a_2}\]


		Naj bo $\theta$ kot med $\vec{a}1$ in x-osjo.

		(slika spet omfg wtffdsdasdas)

		Potem je $\vec{a_1} = ( \|\vec{a_1}\| \cos \theta, \|\vec{a_1}\| \sin \theta , 0)  \implies \vec{a_1}^{\prime} =  (\|\vec{a_1}\| \cos{\phi + \frac{\pi}{2}} , \|\vec{a_1}\| \sin{\phi + \frac{\pi}{2}}, 0)  = ( - \|\vec{a_1}\| \sin \phi, \|\vec{a_1}\| \cos \theta ,0) = (- y_1, x_1,0) \implies P = \vec{a_1}^{\prime} \vec{a_2} = -y_1 x_2 + x_1 y_2$.

		\nt{ Ce bi bil par $\vec{a_1},\vec{a_2}$ negativno orientiran, bi dobili $P = y_1 x_2 - x_1 y_2$.}
	\end{enumerate}
}

Izraz $x_1 y_2 - y_1 x_2$ nam torej pove produkt ploscin paralelograma in orientacije para vektorjev $\vec{a_1},\vec{a_2}$.

Izraz $x_1 y_2 - y_1 x_2$ imenujemo determinanta reda 2 in jo oznacimo z

\[
\begin{vmatrix}
     x_1 & y_1 \\
     x_2 & y_2
\end{vmatrix}
\]

Determinanta je torej produkt ploscine paralelograma in orientacije.


Sledi tudi: Vektorja $(\vec{x_1},\vec{y_1}$ in $(\vec{x_2},\vec{y_2}$ sta linearno odvisna natanko tedaj, ko

\[
\begin{vmatrix}
     x_1 & y_1 \\
     x_2 & y_2
\end{vmatrix}  = 0
\
.\]


\section{Vektorski produkt}

\dfn{Vektorski produkt}{
	Vektorski produkt $\vec{a}$ in $\vec{b}$ je vektor $\vec{a} \times \vec{b}  $, za katerega veljajo naslednje lastnosti:

	\begin{enumerate}
	\item $\vec{a} \times \vec{b}$ je pravokoten na $\vec{a}$ in na $\vec{b}$
		\item	dolzina je ploscina paralelograma, napetega na krajevna vektorja $\vec{a}$ in $\vec{b}$
		\item Trojica $(\vec{a},\vec{b}, \vec{a} \times \vec{b}$ je pozitivno orientirana. To pomeni: ce s konca $\vec{a} \times \vec{b}$ pogledamo na ravnino, ki jo dolocata $\vec{a}$ in $\vec{b}$, potem se $\vec{a}$ zavrti v pozitivni smeri za kot manjsi ali enak $180 \degree$, da dobimo vektor, ki kaze v enako smer kot $\vec{b}$.
	\end{enumerate}
}

\cor{}{
	Ce $\vec{a} \parallel $, potem je $\vec{a} \times \vec{b} = \vec{0}$. Dogovorimo se, da je nicelni vektor vzporeden vsakemu vektorju. Potem velja $ \vec{a} \times \vec{b} = \vec{0} \iff \vec{a} \parallel \vec{b}$.
}

Naj bo $\vec{a_1} = (x_1,y_1,z_1)$ in $\vec{a_2} = (x_2,y_2,z_2)$. Radi bi izracunali komponente vektorja $\vec{a} \times \vec{b}$.
Naj bo $\vec{a} \times \vec{b} = (x_3,y_3,z_3)$. Izracunali bomo $z_3$.

$z_3 = (\vec{a} \times \vec{b}) \vec{k}$

Naj bo $\theta$ kot med $(\vec{a_1} \times \vec{a_2}$ in $\vec{k}$. Potem $z_3 = \| \vec{a_1} \times \vec{a_2}  \| \|\vec{k}\| \cos \theta = \|\vec{a_1} \times \vec{a_2}\| \cos \theta$.

Naj bosta $\vec{a_1}^{\prime}$ in $\vec{a_2}^{\prime}$ pravokotni projekciji vektorjev $\vec{a_1}$ in $\vec{a_2}$ na ravnino $z=0$.


Ocitno je $\vec{a_1}^{\prime} = (x_1,y_1, 0)$ in $\vec{a_2}^{\prime} = (x_2,y_2,0)$. Ogljisca paralelograma med $\vec{a_1}$ in $\vec{a_2}$ so

\[
	(0,0,0), (x_1,y_1,z_1), (x_2,y_2,z_2) \text{in} (x_1 + x_2, y_1 + y_2, z_1 + z_2)
.\]


Ce te tocke projeciramo na ravnino $z=0$, dobimo tocke  $(0,0,0), (x_1,y_1,0), (x_2,y_2,0), (x_1 + x_2, y_1 + y_2,0)$ kar so natanko ogljisca paralelograma, napetega na vektorja $\vec{a_1}^{\prime}$ in $\vec{a_2}^{\prime}$. Po primeru iz skalarnega produkta je ploscina tega projeciranega paralelograma enaka

\[
P^{\prime} = \begin{cases}
	x_1 y_2 - x_2 y_1 ; (\vec{a_1}^{\prime}, \vec{a_2}^{\prime}) \text{  pozitivna orientacija} \\
	-x_1 y_2 + x_2 y_1 ; (\vec{a_1}^{\prime}, \vec{a_2}^{\prime}) \text{  negativna orientacija}
	\end{cases}
\]

Kaksna je zveza med originalnim in projeciranim paralelogramom?



(pol je ena neberjljiva slika holy shit ne vem kaj je gor narisanuasjdhaskdjhasj)



Kot med ravnino, ki jo dolocata $\vec{a_1}$ in $\vec{a_2}$ in ravnino, ki jo dolocata $(\vec{i}, \vec{j})$ (oz. $(\vec{a_1}^{\prime},\vec{a_2}^{\prime})$) je tudi $\theta$.

Naj bo $A$ koncna tocka vektorja  $\vec{a_1}$, $B$ pa koncna tocka vektorja  $\vec{a_2}$, C pa naj bo presecisce premice $OB$ in vzporednice (skozi  $A$) presecisca ravnine, napete med  $\vec{a_1}$ in $\vec{a_2}$ in ravnine $z=0 (\vec{i}, \vec{j})$ ???? (dobr kle nevem glih kaj pise pr men lmao)


Tocke $A,B,C$ projeciramo na ravnino  $z=0$ in dobljene tocke oznacimo z crticami. Daljica  $A^{\prime}C^{\prime}$ in $AC$ sta vzporedni in enako dolgi. Kot med visino na  $AC$ in visina na  $A^{\prime}C^{\prime}$ je tudi 0.

\[
\implies V_{A^{\prime}C^{\prime}} = V_{AC} * |\cos \theta | \\
\]
\[
P_{OA^{\prime}C^{\prime}} = P_{OAC} |\cos \theta |
.\]

(abs. zaradi topih kotov)
Na enak nacin bi dokazali, da je ploscina:

\[
P_{A^{\prime}B^{\prime}C^{\prime}} = P_{ABC} | \cos \theta|
.\]

Ploscina trikotnika $OAB$ je vsota ali razlika ploscin trikotnikov $OAC$ in  $ABC$. Isto velja za projeciran trikotnik.

\[
	\implies P_{OA^{\prime}B^{\prime}} = P_{OAB} | \cos \theta |
.\]

Isto velja za ploscine ustreznih paralelogramov. Torej:

\[
P^{\prime} = \|\vec{a_1} \times \vec{a_2} \| | \cos \theta | \implies z_3 = \|\vec{a_1} \times \vec{a_2}\| \cos \theta = \begin{cases}
	P^{\prime} \text{ce je $\theta$ ostri = + orientacija}\\
	-P^{\prime}  \text{ce je $\theta$ topi = - orientacija}
\end{cases}
= x_1 y_2 - x_2 y_1 =
\begin{vmatrix}
	x_1 & y_1 \\
	x_2 & y_2

\end{vmatrix}
.\]

Na enak nacin dobimo se $x_3 =
\begin{vmatrix}
	y_1 & z_1 \\
	y_2 & z_2

\end{vmatrix}
$
in $y_3 =
\begin{vmatrix}
	z_1 & x_1 \\
	z_2 & z_2
\end{vmatrix}
= -
\begin{vmatrix}
	x_1 & z_1 \\
	x_2 & z_2
\end{vmatrix}
$


(x,y,z) ciklicno zamenjamo.


\[
	(x_1,y_1,z_1) \times (x_2,y_2,z_2) = \left(
	\begin{vmatrix}
		y_1 & z_1 \\
		y_2 & z_2
	\end{vmatrix}
	, -
	\begin{vmatrix}
		x_1 & z_1 \\
		x_2 & z_2
	\end{vmatrix}
	,
	\begin{vmatrix}
		x_1 & y_1 \\
		x_2 & y_2
	\end{vmatrix}
	\right)
.\]

Determinante reda 3. je definirana s predpisom:
\[
\begin{vmatrix}
	a_1 & b_1 & c_1 \\
	a_2 & b_2 & c_2 \\
	a_3 & b_3 & c_3 \\

\end{vmatrix}
= a_1
\begin{vmatrix}
	b_2 & c_2  \\
	b_3 & c_3

\end{vmatrix}
- b_1
\begin{vmatrix}
	a_2 & c_2 \\
	a_3 & c_3

\end{vmatrix}
+ c_1
\begin{vmatrix}
	a_2 & b_2 \\
	a_3 & b_3

\end{vmatrix}
\implies (x_1, y_1, z_1) \times (x_2, y_2, z_2) =
\begin{vmatrix}
	\vec{i} & \vec{j} & \vec{k} \\
	x_1 & y_1 & z_1 \\
	x_2 & y_2 & z_2
\end{vmatrix}
.\]


Lastnosti vektorskega produkta:
\begin{enumerate}
	\item Anti-komutativnost : $\vec{a} \times \vec{b} = - \vec{b} \times \vec{a}$
	\item Distributivnost: $\vec{a} \times (\vec{b} + \vec{c}) = \vec{a} \times \vec{b} + \vec{a} \times \vec{c} \text{ali} (\vec{b} + \vec{c}) \times \vec{a} = \vec{b} \times \vec{a} + \vec{c} \times \vec{a}$
	\item Homogenost: $(\alpha \vec{a}) \times \vec{b} = \alpha (\vec{a} \times \vec{b})$
\end{enumerate}


\ex{}{
	Izracunaj ploscino trikotnika z ogljisci $A(1,0,2), B(2,2,0), C(3,-2,1)$.
	Po definiciji vektorskega produkta je  $\vec{p} = \frac{1}{2} \| \vec{AB} \times \vec{AC}\|$.

	$\vec{AB} = (1,2,-2)$, $\vec{AC} = (2,-2,-1)$.
	$\vec{AB} \times \vec{AC} =
        \begin{vmatrix}
		\vec{i} & \vec{j} & \vec{k} \\
		1 & 2 & -2 \\
		2 & -2 & -1
        \end{vmatrix}
        = \left(
	\begin{vmatrix}
		2 & -2 \\
		-2 & -1
	\end{vmatrix}
	,-
	\begin{vmatrix}
		1 & -2 \\
		2 & -1
	\end{vmatrix}
	\right)
	= (-6,-3,-6)$

	$p = \frac{1}{2} \| (-6,-3,-6) \|  = \frac{1}{2} \sqrt{36 + 9 + 36} = \frac{1}{2} 9 = \frac{9}{2}$.

}



\section{Mesani produkt}

Mesani produkt vektorjev $\vec{a}, \vec{b}, \vec{c}$ je stevilo $(\vec{a} \times \vec{b}) \vec{c}$. Oznacili ga bomo z $\left[\vec{a},\vec{b},\vec{c}\right]$.


\[
\left[\vec{a},\vec{b},\vec{c}\right] = (\vec{a} \times \vec{b}) \cdot \vec{c}
.\]

(Ok zdj sm opazu da sm posazu pisat cdot tk da shikatanai).


Naj bo $\vec{a_1} = (x_1, y_1, z_1), \vec{a_2} = (x_2,y_2,z_2), \vec{a_3} = (x_3,y_3,z_3)$.

\[
	(\vec{a_1} \times \vec{a_2}) \cdot \vec{a_3} =
	\begin{vmatrix}
		\vec{i} & \vec{j} & \vec{k} \\
		x_1 & y_1 & z_1 \\
		x_2 & y_2 & z_2
	\end{vmatrix} \cdot
	\left( x_3,y_3,z_3 \right)
	=
	\left(
        \begin{vmatrix}
		y_1 & z_1 \\
		y_2 & z_2
        \end{vmatrix}
	, -
	\begin{vmatrix}
		x_1 & z_1 \\
		x_2 & z_2
	\end{vmatrix}
	,
	\begin{vmatrix}
		x_1 & y_1 \\
		x_2 & z_2
	\end{vmatrix}
	\right) \cdot (x_3,y_3,z_3) =\]
	\[
	= x_3
	\begin{vmatrix}
		y_1 & z_1 \\
		y_2 & z_2
	\end{vmatrix}
	, -y_3
	\begin{vmatrix}
		x_1 & z_1 \\
		x_2 & z_2

	\end{vmatrix}
	, z_3
	\begin{vmatrix}
		x_1 & y_1 \\
		x_2 & y_2

	\end{vmatrix}
	=
	\begin{vmatrix}
		x_3 & y_3 & z_3 \\
		x_1 & y_1 & z_1 \\
		x_2 & y_2 & z_2
	\end{vmatrix}
.\]



Geometrijska interpretacija mesanega produkta: paralelepepid je geometrijsko telo, doloceno s tremi cetvericami paroma vzporednih robov. (nagnjen kvader).


Prostornina $V$ parelepepida je produkt osnovene ploske $p$ in visine $v$. Ppp. je dolocen s tremi vektorji (slike galore) :

 $V = p v$


Kot med $\vec{c}$ in visino oznacino z $\delta$.

 \[
V = p v = \|\vec{a} \times \vec{b}\| \cdot v = \|\vec{a} \times \vec{b}\| \|\vec{c}\| |\cos \delta|
.\]
(abs vrednost ce vektor c kaze navzdol).
(torej + ce (a,b,c) pozitivno orientirani) ali - v nasprotnem primeru).

$ \implies (\vec{a} \times \vec{b}) \cdot \vec{c} \text{  ; } \left( \vec{a}, \vec{b}, \vec{c} \right) $ kadar so $\left( \vec{a},\vec{b},\vec{c} \right) $ pozitivno orientirani in $- \left( \vec{a} \times \vec{b} \right) \cdot \vec{c}$ sicer.


Mesani produkt $\left[ \vec{a},\vec{b},\vec{b}  \right] $ je torej produkt prostornine paralepeptida, napetega na $\vec{a},\vec{b}$ in $\vec{c}$ ter orientacije vektorjev.

\cor{}{
\[
	\left[ \vec{a},\vec{b},\vec{c}  \right] = \left[ \vec{c},\vec{a},\vec{b}  \right] = \left[ \vec{b},\vec{c},\vec{a} \right] =
	-\left[ \vec{b},\vec{a},\vec{c}  \right] = -\left[ \vec{c},\vec{b},\vec{a}  \right]  = -\left[ \vec{a},\vec{c},\vec{b}  \right]
.\]

Ciklicno menjanje ne spremeni orientacije, menjava dveh elementov pa spremeni.
}


\cor{}{

\[
	\left[ (x_1,y_1,z_1), (x_2,y_2,z_2), (x_3,y_3,z_3) \right]=
	\begin{vmatrix}
		x_1 & y_1 & z_1 \\
		x_2 & y_2 & z_2 \\
		x_3 & y_3 & z_3

	\end{vmatrix}
.\]
}

\cor{}{
\[
	\begin{vmatrix}
		x_1 & y_1 & z_1 \\
		x_2 & y_2 & z_2 \\
		x_3 & y_3 & z_3

	\end{vmatrix} = 0
\]

natanko tedaj, ko so vektorji $(x_1,y_1,z_1), (x_2,y_2,z_2),(x_3,y_3,z_3)$ linearno odvisni.
}

Lastnosti mesanega produkta:

\begin{enumerate}
	\item Distributivnost v vseh treh faktorjih:
		$\left[ \vec{a_1} + \vec{a_2}, \vec{b}, \vec{c} \right] =  \left[  \vec{a_1},\vec{b}, \vec{c}  \right] = \left[  \vec{a_2} , \vec{b}, \vec{c} \right]  $
		enako za ostala dva vektorja.
	\item Homogenost v vseh treh faktorjih: $\left[  \alpha \vec{a}, \vec{b} , \vec{c}  \right] = \left[  \vec{a}, \alpha \vec{b}, \vec{c}  \right] = \left[  \vec{a}, \vec{b} , \alpha \vec{c}  \right]   = \alpha \left[  \vec{a} , \vec{b} , \vec{c}  \right] $. Preverimo z racunom in upostevamo lastnosti skalarnega in vektorskega produkta.
\end{enumerate}


\section{Dvojni vektorski produkt}




\end{document}
